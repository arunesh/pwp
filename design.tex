\section{Overall Design}
\textsf{Picolo} is a homogeneous network of nodes running a p2p database software. The network has an open participation model - any node that runs \textsf{Picolo}'s software can freely join the network. It may elect to become a \textsf{storage provider} (\cref{sec:participants}) by depositing \textsf{PINT}s (Picolo Network Tokens) and become eligible for earning rewards or it can take on the role of a data provider or a consumer. Storage providers power the network and are organized into \textsf{clusters}. Each cluster serves a \textsf{storage consumer}, typically a dapp and consists of \textsf{shards}. Each shard is made up of small pieces of data sets defined by a \textsf                                                                                                                                                                                                                                                                                                                                                                                                                                                                                                                                                                                                                                                                                                                                                                                                                                                                                                                                                                                                                                                                                                                                                                                                                                                                                                                                                                                                                                                                                                                                                                                                                                                                                                                                                                                                                                                                                                                                                                                                                                                                                                                                                                                                                                                                                                                                                                                                                                                                                                                                                                                                                                                                                                                                                                                              {key range}. Each such key range is replicated for durability and availability where consensus among replicas is achieved by running a paxos based algorithm. Clusters are horizontally scalable by adding more nodes and splitting data into new shards. Storage consumers can directly connect to clusters by-passing the need for DHT based lookup, improving latencies.
\newline\newline
Nodes are connected to each other via a DHT overlay network and a prefix-based routing algorithm that locates content using a content-based addressing mechanism. The overlay network includes methods for discovery of
content/nodes, fast and opportunistic routing and handle node dynamics such as failures and node additions. The
routing framework includes optimizations which allow nodes that belong to a cluster to cache localized
route information. Cached links allow for optimizations that provide an O(1) lookup latency for most queries. Cached content and links use what we call soft-state publishing to allow for faster convergence in
the face of node churn. A ladder based Trust and Governance model which is built using a combination of earned trust
and stake, ensures proper decentralized operation of the network in the face of certain types of malicious intent. This
design is discussed in detail in \cref{net}.

\subsection{Incentive compatibility}
\textsf{Picolo} is an incentive compatible network. All participants (\cref{sec:participants}) in the network are assumed to be rational and acting in their best interest and may not necessarily be interested in the overall benefit of the network. Hence we employ a mechanism design that incentivizes participants to work towards the proper functioning of the network while keeping individual interests in mind. For instance, we enhanced the normal pings that nodes in a database cluster send to each other for health checking with requests for small amounts of randomly sampled data a node is responsible for. If a node fails this check, its security deposit is slashed. Hence, storage consumers can have reasonable confident that their data is being reliably stored. At the same time, storage providers are paid for storing and serving data correctly, hence it is also in their best interest to not corrupt or delete consumer's data.

\subsection{Participants} \label{sec:participants}
There are four types of participants in the network:
\begin{enumerate}
	\item Storage providers
	\item Storage consumers
	\item Data providers
	\item Data consumers \newline
\end{enumerate}
\textbf{Storage providers}: Storage providers provide compute resources that power the network such as CPU, memory and disk. In our current architecture, these resources are on the same machine. While its possible to separate the ``server'' part from ``storage'' part as in databases with NAS (network attached storage), in a p2p world with commodity network links, such a design introduces unacceptable latencies. Storage providers are compensated for powering the network by storage consumers. 
\newline\newline
\textbf{Storage consumers}: \DJ app developers that need to store structured data fall into this category. They can simply use the database functionalities provided by the network without the need to perform any database administration tasks. In this sense, the network gives them the same ease of use as a cloud hosted database provider but at a much cheaper rate.
\newline\newline
\textbf{Data providers}: These are people who want to share their data with the world. This could be highly domain specific data that doesn't lend itself well for sharing using existing tools. As an example, one might write a program that reads the state of a blockchain like Ethereum and store it in a query-able format on \textsf{Picolo}. They can then sell access to this data. More specifically, one such program might entail observing blockchain state to draw insights on how dapps of a particular category are being used. Storage consumers above can also be data providers; for example an end user using a dapp that stores data on \textsf{Picolo} and puts data control in the hands of the user enables them to become a data provider for other dapps. 
\newline\newline
\textbf{Data consumers}: These are people/applications/nodes interested in data shared by data providers. Future of application building does not include hoarding user data in silos, instead innovative applications will be built using data commonly available or shared at a single location. This is the idea behind the paradigm of ``code comes to data'' where users solely control their data and allow applications to provide services by selectively giving access to their data.


