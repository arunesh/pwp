\documentclass[preprint,10pt]{elsarticle}
\usepackage{etoolbox}
\makeatletter
\patchcmd{\ps@pprintTitle}{\footnotesize\itshape
       Preprint submitted to \ifx\@journal\@empty Elsevier
       \else\@journal\fi\hfill\today}{06.2018}{}{}
\makeatother

\usepackage[margin=2.5cm]{geometry}

\newcommand{\fscale}[1]{#1\linewidth}
\newcommand{\figref}[1]{Fig.~\ref{#1}}

%-----------------------------------------------------------------------------
%  Colors
%-----------------------------------------------------------------------------
\usepackage{pagecolor}
\definecolor{linen}{RGB}{240,240,230}
\definecolor{fwhite}{RGB}{255,250,240}
\definecolor{oldlace}{RGB}{253,245,230}
\definecolor{awhite}{RGB}{250,235,215}
\definecolor{pwhip}{RGB}{255,239,213}
\definecolor{peach}{RGB}{255,218,185}
\definecolor{ivory}{RGB}{255,255,240}
\definecolor{seashell}{RGB}{	255,245,238}

\usepackage{sectsty}
\sectionfont{\Large}
\subsectionfont{\large}

\usepackage{amsmath}
\newcommand{\mr}[1]{\mathrm{#1}}  % Roman font for the math mode
\newcommand{\mi}[1]{\mathit{#1}}  % Italic font for the math mode
\newcommand{\mc}[1]{\mathcal{#1}} % Caligrafic (script) font for the math mode
\newcommand{\ms}[1]{\mathsf{#1}}  % Sans font for the math mode (vectors and matrices)
\newcommand{\mb}[1]{\mathbf{#1}}  % Bold font for the math mode (vectors and matrices)

\usepackage{mathtools}
\DeclarePairedDelimiter\ceil{\lceil}{\rceil}
\DeclarePairedDelimiter\floor{\lfloor}{\rfloor}

\usepackage{amsthm}
\theoremstyle{definition}
\newtheorem{definition}{Definition}[section]
\newtheorem{remark}{Rule}[section]

\newtheorem{property}{Property}

\newtheorem{theorem}{Theorem}

\usepackage{cleveref}
\crefformat{section}{\textbf{\S#2#1#3}} % see manual of cleveref, section 8.2.1
\crefformat{subsection}{\textbf{\S#2#1#3}}
\crefformat{subsubsection}{\textbf{\S#2#1#3}}

\renewcommand{\cref}[1]{\textbf{\S\ref{#1}}}

\usepackage{graphicx}
\usepackage{subfigure}
\graphicspath{{./fig/}}
%% The amssymb package provides various useful mathematical symbols
\usepackage{amssymb}

\usepackage[titletoc]{appendix}
\usepackage[T1]{fontenc}

\usepackage{xcolor, soul}
\sethlcolor{oldlace}

\usepackage{lineno}
\usepackage{url}
\bibliographystyle{unsrt}

\patchcmd{\abstract}{Abstract}{Summary}{}{}

\begin{document}

\begin{frontmatter}

%% Title, authors and addresses

\title{Picolo: Fast p2p open database network\tnoteref{title1}}
\tnotetext[title1]{Version 1.2}

\author{Picolo Labs\corref{auth1}}
\cortext[auth1]{https://picolo.network}
\address{San Francisco, California}

\begin{abstract}
%% Text of abstract
Picolo is a fast, scalable, fully decentralized, globally distributed transactional database network for blockchain based applications. It is a p2p network with an open participation model where any node can freely join and leave at will. Nodes discover each other in a decentralized manner via DHTs where each node maintains just enough routing information that will enable it to discover the topology of the whole network. It uses a probabilistic replication framework on top of DHTs to achieve an O(1) lookup latency in the average case. It also adds a layer on top of DHTs that allows locating nodes by geographic regions.
\newline
\newline
All nodes in the network are symmetric - they all run the same database software that provides a number of desirable features. Clients can interact with them via a generic SQL interface that has familiar methods such as selections, aggregations and projections. Nodes typically are part of a replica group (of a small size like 5) and the network is made up of many such groups. Nodes in a replica group are responsible for the same data where mutations to it are synchronously replicated to all replicas. Mutations are applied in the order seen by nodes where order is determined by timestamps that are a combination of physical and logical clocks. This gives Picolo the property of external consistency - the strongest consistency property a database can achieve. Distributed transactions across nodes are also made possible by using these hybrid timestamps. Automatic data sharding is supported when it grows too big or for load balancing purposes. Queries posed to the network can be run across all nodes with vastly differing schemas. Overtime, semantically similar schemas are logically grouped together for faster processing and richer results. There is a notion of \textsf{user-controlled} and \textsf{app-controlled} schemas that aids in achieving data sovereignty. Fine grained access control rules to data can be defined via a declarative language which are then enforced via encryption.
\newline
\newline
There are four types of participants in the network: \textsf{storage providers}, \textsf{storage consumers}, \textsf{data providers} and \textsf{data consumers}. Storage providers join the network to offer their spare computing resources like disk space, cpu and memory in exchange for cryptographic tokens. Consumers can pay providers with these tokens (among other ways) and use the resources provided to satisfy their needs. Since the network is open for anyone to participate, consumers need some safety guarantees before they can reliably store their data on untrusted nodes. So providers are required to put up security deposits and the network is designed to be tolerant to their byzantine behavior, to punish malicious nodes by slashing their deposits and to reward honest nodes for diagnosing and reporting byzantine behavior. A mechanism called \textsf{Proof-of-Query} is used for the diagnosis and deposit slashing is enforced by a smart contract system.
\end{abstract}

\begin{keyword}
	\textsf{decentralization \sep blockchain \sep ethereum \sep database \sep sql \sep access control \sep secret sharing \sep p2p \sep tokens \sep incentives \sep mechanism design \sep byzantine faults \sep data sharing}
\end{keyword}

\end{frontmatter}

\newpage
\tableofcontents
\newpage
%-----------------------------------------------------------------------------
%  INTRODUCTION
%-----------------------------------------------------------------------------
\section{Introduction}\label{Sect:Introduction}
We are in the midst of building a new Internet. Blockchain networks like Ethereum have popularized the idea of unstoppable, owner less applications that are run on an open network of untrusted but incentivized nodes without the oversight of a central authority. An increasing number of these decentralized applications (\DJ apps) are being created everyday with evolving data storage needs. The first generation of these dapps stored their data entirely on a blockchain itself while the current ones are storing it on decentralized file storage systems like IPFS with just hashes of the data stored on the blockchain. Although this method works for dapps with simple storage needs like the need for storing data as a blob, it is very limiting for dapps with more complex requirements such as the need to store data at a more fine grained level and efficiently querying it. A popular option is then to use a cloud hosted database (ex: Google’s Cloud SQL) but that turns a dapp into a non-dapp by introducing centrality into the system. \newline\newline
In this paper we introduce \textsf{Picolo}, an open database network that combines the benefits of a traditional database and p2p software. It is the world's first NewSQL $decentralized$ database system that offers features such as a SQL interface, distributed transactions, external consistency \cite{External_Consistency}, lock-free reads and snapshot reads. It is an open network that allows anyone with spare computing power and disk space to join the network and get rewarded for hosting and serving structured data.
\newline\newline
Features that set \textsf{Picolo} apart from other decentralized databases:
\begin{itemize}
	\item Transactions can be applied across rows, columns and tables across nodes
	\item Client controlled replication and data placement
	\item Support for storage of typed data
	\item Support for semi-relational structure for tables
	\item Configurable backups and restore mechanisms
	\item Allows for verifiable transaction logs
	\item Data poisoning detection
	\item Distributed query processing
	\item Fine grained access control to data
	\newline
\end{itemize}
Rest of the paper is structured as follows: in section 2, we discuss some related work. Section 3 presents the overall design of the system, section 4 presents the network subsystem, section 5 presents the database subsystem and section 6 presents our approach to dealing with problems that arise in a network made up of untrusted nodes and discusses an incentive mechanism to make them behave as per system's needs. In section 7, we propose a message exchange protocol in the same vein as http but for relational data sharing. Section 8 concludes the paper.

%-----------------------------------------------------------------------------
%  BACKGROUND SECTION
%-----------------------------------------------------------------------------
\input{background}

%-----------------------------------------------------------------------------
%  OVERALL DESIGN SECTION
%-----------------------------------------------------------------------------
\input{design}

%-----------------------------------------------------------------------------
%  NETWORK SUB-SYSTEM SECTION
%-----------------------------------------------------------------------------
\input{network}


%-----------------------------------------------------------------------------
%  DATABASE SUB-SYSTEM SECTION
%-----------------------------------------------------------------------------
\input{db}

\input{crypto}

%-----------------------------------------------------------------------------
%  CONCLUSIONS
%-----------------------------------------------------------------------------
\section{Conclusions and Future work}
We described a database network that serves as the backbone for web 3.0 data. The network is made up of homogeneous nodes that self-organize into a p2p network based on DHTs and run a database software that provides features such as SQL interface, external consistency, auto-sharding of data and self-replication in case of failures. The network is incentive-compatible and has mechanisms in place for rewarding desirable behavior and punishing malicious intent. The network can be used by blockchain apps to store their contract state or normal apps to store generic application data and put the data owner in complete control. We discussed mechanisms to ensure data confidentiality, integrity, availability and a scheme to provide fine grained access control. We listed various attacks that are possible on the network and their respective mitigation/prevention measures.
\newline\newline
We leave the practical construction and performance analysis of attribute-based encryption scheme for a future paper. Methods to detect access control leakage, more robust defenses against DDos and Sybil attacks are left as a future exercise. A more generic access control framework based on functional encryption \cite{functional_encryption} that promises increased privacy and seems to be a step towards the holy grail of fully homomorphic encryption \cite{FHE} will also be explored in the future.
%-----------------------------------------------------------------------------
%  BIBLIOGRAPHY
%-----------------------------------------------------------------------------
\section{References}
\bibliography{./bib/picolo.bib}

\include{appendix}

\end{document}
